
\subsection*{\textbf{Question 4 - Summary}}
\begin{quote}

\textbf{Summary}
\begin{quote}
The current sub-section contains the summary of the code used for assignment 4. This includes the part of the main file that executes the subquestions, the relevant imports, the functions to calculate the linear growth factor and the functions used to generate and sort the matrices (tensor).

\end{quote}


\textbf{Code - Assignment}

\begin{quote}
The code with the main function and the function \texttt{gen\_complex}, which is called in 4c and 4d. 
\label{CODE:MAIN4}
\lstinputlisting[firstline=0,lastline=34]{./Code/assigment_4.py}
\end{quote}

\textbf{Code - Linear growth} \\
\begin{quote}
The code which calculates the linear growth factor and the derivative of the liner growth factor.
\label{CODE:h4}
\lstinputlisting{./Code/mathlib/helpers4.py}
\end{quote}

\textbf{Code - Tensor/matrix}
\begin{quote}
The code containing all the function sued to create the matrices (tensors) and make then symmetric.  
\lstinputlisting{./Code/mathlib/misc.py}
\label{CODE:misc}
\end{quote}

\textbf{Code - Romberg}
\begin{quote}
The code for romberg integration
\lstinputlisting{./Code/mathlib/integrate.py}
\label{CODE:misc}
\end{quote}


\end{quote}

\newpage

%\textbf{Code - output } 
%\begin{quote}
% The code that produces the output.
%\lstinputlisting{./code/assigment1_a.py}
%\end{quote}

%\textbf{Code - helper } 
%\begin{quote}
%The code for the Poisson distribution and the factorial function.  
%\lstinputlisting[firstline=2,lastline=46]{./code/mathlib/utils.py}
%\end{quote}


%\textbf{Output}
%\begin{quote}
%The output produced by \textsf{/code/assigment1\_ a.py} 
%\lstinputlisting{./output/assigment1_a_out.txt}
%\end{quote}
\newpage











