
\subsection*{\textbf{Question 4.d}}
\begin{quote}

\textbf{Problem}
\begin{quote} 
Generate initial conditions for a three dimensional box to do a N-body simulation, make initial conditions for $64^3$ particles starting at redshift $z = 50$. Besides this make 3 sperate movies of a slice of thickness 1/64th of your box at its center, make a slice for $x-y$, $x-z$,$y-z$. Again make a movie of at least 3 seconds with at least 30 frames per second. Finally plot the position and momentum of the first 10 particles along the $z-direction$ vs $a$.
\end{quote}

\textbf{Solution} 
\begin{quote}
The method in which the displacement vector is similar to the method in 4c. This thus means that first a 3D matrix (tensor) is created in k-space with complex values based on the given power law. The matrix is next given the correct hermitian symmetry, which is done by an extended version of the algorithm explained in question 2. The symmetric matrix (tensor) is now used to calculate the components $s_x$, $s_y$ and $s_z$. This is  done by multiplying the terms in the tensor with the correct wavenumbers and $i$. In the end this results in three matrices that need to be inverse fourier transformed to obtain the components of $\textbf{S}$. The three matrices can however not directly be inverse fourier transformed as the multiplication with the wavenumbers breaks the symmetry in the nyquest planes. The symmetry is again only broken by a minus sign and is first corrected before doing the IFFT. The corrected matrices are then used to calculate $s_x$, $s_y$ and $s_z$. 
\\
The code that uses the displacement vector to creates the simulation and the plots for the first 10 particles can found below.

\end{quote}

\textbf{Code -plots}:
\begin{quote}

The code that creates the movie and the plots of the first 10 particles. 
\lstinputlisting[firstline = 27, lastline=158]{./Code/assigment_4.py}
\end{quote}


\textbf{Plots - particles}
\begin{quote}
\begin{figure}[!ht]
\centering
\includegraphics[width=14cm, height=9.5cm]{./Plots/4d_pos.pdf}
\caption{The z-positions of the first 10 particles against the scale factor.}
\end{figure}
\newpage
\begin{figure}[!ht]
\centering
\includegraphics[width=14cm, height=9.5cm]{./Plots/4d_momentum.pdf}
\caption{The z-component of the momentum of the first 10 particles against the scale factor. }
\end{figure}
\end{quote}



\end{quote}




