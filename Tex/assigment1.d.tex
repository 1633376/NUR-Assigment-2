
\subsection*{\textbf{Question 1.c}}
\begin{quote}

\textbf{Problem}
\begin{quote}
Write a code that does the Kuiper’s test on your random numbers (see tutorial
8) and make the same plot as for the KS-test.
\end{quote}

\textbf{Solution} 
\begin{quote}
The implementation of the Kuiper test does simmilar to the KS-test require a numerical approximation of the kuiper statistics,

%KS-test is in general straight forwards. There are however two points of interest that needs to be discussed. The first point is the implementation of the CDF for the KS-statistic and the second point is the implementation of the CDF for the normal distribution.
\\

\textbf{(1) CDF KS-statistic }
\begin{quote}
The p-value produced by the KS-tests requires the evaluation of the CDF for the KS-test statistic,
\begin{equation}
P_{KS}(z) = \frac{2\sqrt{\pi}}{z} \sum_{j=1}^{\infty} \exp\left(- \frac{(2j-1)^2+\pi^2}{8z^2} \right)
\end{equation}

This infinite sum needs to be numerically approximated in order to perform the KS-test. The chosen approximation in the implementation of the KS-test for  the sum is taken from ... who states that the sum can be approximated by, % approximation can according to .. be approximated by, 
\begin{equation}
P_{KS}(z) \approx
\begin{cases}
\frac{\sqrt{2 \pi}}{z} \left[ \left( e^{-\pi^2/(8z^2)} \right)+^{9} + \left( e^{-\pi^2/(8z^2)} \right) \left( e^{-\pi^2/(8z^2)} \right)^{25} \right] &\quad \text{for $ z < 1.18$}\\
1 -2 \left[ \left(e^{-2z^2}\right) - \left(e^{-2z^2}\right)^4 + \left(e^{-2z^2}\right)^9 \right]  &\quad \text{for $ z >= 1.18$}
\end{cases}
\end{equation}
\end{quote}

\textbf{(2) CDF normal distribution}
\begin{quote}
The CDF of the normal distribution is needed in order to perform the KS-test under the null hypothesis that the data follows a normal distribution. The CDF of the normal distribution can in general be written as,
\begin{equation}
\Phi\left( \frac{x- \mu}{\sigma}\right) = \frac{1}{2} \left[ 1 + \text{erf} \left( \frac{x - \mu}{\sigma\sqrt{2}} \right) \right]
\end{equation}

where the erf is given by,
\begin{equation}
\text{erf}(x)  \frac{2}{\sqrt{\pi}} \int_0^{x} e^{-t^2} dt
\end{equation}

The integral of the erf function lacks a closed form and therefore also needs to be numerically approximated. The chosen approximation is taken from ..., who states that the function can be 
approximated by,

\begin{equation}
\text{erf}(x) \approx 1- (a_1t+a_2t^2 + ... + a_5t^5)e^{-x^2} \text{x} \quad t = \frac{1}{1+px}
\end{equation}

where $p  =0.3275911$, $a_1 =  0.254829592$, $a_2 = -0.284496736$, $a_3 = 1.421413741$, $a_4 =  -1.453152027$, $a_5 = 1.061405429$.
\end{quote}

The implementation of the KS-tests and using it to test the null hypothesis that the samples follow a normal distribution is with the above approximations implemented. The code for the KS-test and the approximations for the CDF's is located in the file .... at page, as this file is threaded as module. The code for the creation of the plot is given below.  
\end{quote}

\newpage
\textbf{Code - Plots}

\begin{quote}
The code for generating the two plots.
\lstinputlisting[firstline=117,lastline=169]{./Code/assigment1.py}
\end{quote}
\newpage

\textbf{Code - Output plot(s)}
\begin{quote}
\begin{figure}[!ht]
\centering
\includegraphics[width=12cm, height=7.5cm]{./Plots/1_plot_ks_test_self.pdf}
\caption{TODO}
\end{figure}

\begin{figure}[!hb]
\centering
\includegraphics[width=12cm, height=7.5cm]{./Plots/1_plot_ks_test_self_scipy.pdf}
\caption{TODO}
\end{figure}

\end{quote}



\end{quote}

%\textbf{Code - output } 
%\begin{quote}
% The code that produces the output.
%\lstinputlisting{./code/assigment1_a.py}
%\end{quote}

%\textbf{Code - helper } 
%\begin{quote}
%The code for the Poisson distribution and the factorial function.  
%\lstinputlisting[firstline=2,lastline=46]{./code/mathlib/utils.py}
%\end{quote}


%\textbf{Output}
%\begin{quote}
%The output produced by \textsf{/code/assigment1\_ a.py} 
%\lstinputlisting{./output/assigment1_a_out.txt}
%\end{quote}
\newpage











