\section*{\textbf{1 - Normally distributed pseudo-random numbers} \hrule} 



\subsection*{\textbf{Question 1.a}}
\begin{quote}

\textbf{Problem}
\begin{quote}Write a random number generator that returns a random floating-point number between 0 and 1. At minimum, use some combination of an MWC and a 64-bit XOR-shift. Plot a sequential of random numbers against each other in a scatter plot ($x_{i+1}$ vs $x_{i}$) for the first 1000 numbers generated. Also plot the value of the random numbers for the first 1000 numbers vs the index of the random number, this mean the x-axis has a value from 0 through 999 and the y-axis 0 through 1). Finally, have your code generate 1,000,000 random numbers and plot the result of binning these in 20 bins 0.05 wide. 
\end{quote}

\textbf{Solution} 

\begin{quote}
bleh
\end{quote}
\end{quote}

%\textbf{Code - output } 
%\begin{quote}
% The code that produces the output.
%\lstinputlisting{./code/assigment1_a.py}
%\end{quote}

%\textbf{Code - helper } 
%\begin{quote}
%The code for the Poisson distribution and the factorial function.  
%\lstinputlisting[firstline=2,lastline=46]{./code/mathlib/utils.py}
%\end{quote}


%\textbf{Output}
%\begin{quote}
%The output produced by \textsf{/code/assigment1\_ a.py} 
%\lstinputlisting{./output/assigment1_a_out.txt}
%\end{quote}
\newpage











