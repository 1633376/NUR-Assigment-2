\documentclass[a4paper,10pt]{article}
\usepackage[utf8]{inputenc}
\input{./Tex/included_packages}

\renewenvironment{abstract}
 { \vspace*{0.3cm} \textbf{\abstractname} \vspace{0.1cm} \\ \ignorespaces}
 {\par\medskip \vspace{0.1cm}}

\setlength{\parindent}{0em}

\setlength{\textheight}{25.7cm}
\setlength{\textwidth}{18cm}
\setlength{\unitlength}{1mm}
\setlength{\topskip}{2truecm}

\topmargin 260mm \advance \topmargin -\textheight
\divide \topmargin by 2 \advance \topmargin -1in
\headheight 0pt \headsep 0pt \leftmargin 210mm \advance
\leftmargin -\textwidth
\divide \leftmargin by 2 \advance \leftmargin -1in
\oddsidemargin \leftmargin \evensidemargin \leftmargin
\parindent=0pt
\frenchspacing


%opening
\title{\textbf{Numerical Recipes for Astrophysics \\ Solutions hand-in assignment-2}}
\author{Luther Algra - s1633376}

\begin{document}

\maketitle

\hrule
\begin{abstract}
The current document contains the solutions for the second hand-in assignment of Numerical Recipes. Each main question 1, 2, 3, ..., 7 is given its own section and contains a subsection for each sub-question (1.a, 1.b, ..., 1.f). A main question always ends with a final subsection that contains two segments of code. The first segment contains the full code of the program that executes the sub-questions. The second segment contains the shared modules (if any) used by the sub-question. A sub-question its self always starts with a short summary of the question that needs to be answered followed by an explanation of how the problem is solved. Next, the code and its output are provided. Finally the output is discussed if relevant. 


\end{abstract}
\hrule
\vspace{0.5cm}


\section*{\textbf{1 - Normally distributed pseudo-random numbers} \hrule} 



\subsection*{\textbf{Question 1.a}}
\begin{quote}

\textbf{Problem}
\begin{quote}Write a random number generator that returns a random floating-point number between 0 and 1. At minimum, use some combination of an MWC and a 64-bit XOR-shift. Plot a sequential of random numbers against each other in a scatter plot ($x_{i+1}$ vs $x_{i}$) for the first 1000 numbers generated. Also plot the value of the random numbers for the first 1000 numbers vs the index of the random number, this mean the x-axis has a value from 0 through 999 and the y-axis 0 through 1). Finally, have your code generate 1,000,000 random numbers and plot the result of binning these in 20 bins 0.05 wide. 
\end{quote}

\textbf{Solution} 


\begin{quote}
The state of the random number generator is updated by first performing a 64-bit XOR-shift on the current state. Next, a modified version of the 64-bit XOR-shift output is given to the MWC algorithm. The modified XOR-shifts output given to the MWC algorithm is the output of the 64 XOR-shift with the last 32 bits put to zero. This is done by performing the 'AND' operation with the maximum value of an unsigned int 32. This modification was performed as the  MWC algorithm expects as input a 64-bit unsigned integer with a value between  $0 < x < 2^{32}$.  The output of the MWC is finally XORd with the unmodified output of the 64-bit XOR-shift. The result is set as new state.
\\
The first 32 bits of the new state are used to provide a random value, as the output of the MWC algorithm only contains 32 significant bits. This random value is obtained by performing the 'AND' operation between the seed and the maximum value of an unsigned int 32. The resulting value is then divided by the maximum value of an uint32 to obtain a value between 0 and 1.
\\
The code for the random number generator can be found at the end of this section, as it is treated as a shared module. The code for generating the plots and the created plots can be found below. The code does not only print the random seed, but also prints the maximum and minimum number of counts for the binned 1,000,000 values. These values are referred to in the description of the plot that displayed the uniformness.
\end{quote}
\newpage

\textbf{Code - Plots}


% consists of the code that initializes the random number generator and calls the function.

\begin{quote}
The code for generating the plots. The used imports and the initialization of the random number generator are not explicit shown in this piece of code, but can be found on page .. where the full code is shown that contains all sub-questions together. The code for the random number generator can be found on page,... as it is treated as a shared module.

\lstinputlisting[firstline=25,lastline=59]{./Code/assigment_1.py}
\end{quote}
\end{quote}

\textbf{Code - Output text } 
\begin{quote}
The text output produced by the code:
\lstinputlisting[firstline=0,lastline=1]{./Output/assigment1_out.txt}
\end{quote}
\newpage

\textbf{Code - Output plots}
\begin{quote}

\begin{figure}[!ht]
\centering
\includegraphics[width=12cm, height=7.5cm]{./Plots/1_plot_against.pdf}
\caption{A plot of random number $x_{i+1}$ against random number  $x_{i}$ for the first 1000 random uniforms produced by the random number generator. A good random number generator should produce a homogeneous plot without many (large) empty spots. In the above plot large empty spots appear to be absent, which suggests that the random number generator passes this test. }
\end{figure}

\begin{figure}[!hb]
\centering
\includegraphics[width=12cm, height=8.0cm]{./Plots/1_plot_index.pdf}
\caption{The first 1000 random uniform numbers produced by the random number generator (RNG) against their index. A good random number generator should not have large wide gaps  (e.g when moving from index 400 to 450 it should not only produce values larger than 0.8, which would leave a wide gap). In the plot these gaps appear to be absent. The average value produced by the RNG should furthermore be 0.5. This corresponds to rapidly moving up and down  the line $ y = 0.5$. In the plot this should, result in a 'dense' region (less white) around the line $y = 0.5$. It can indeed be seen that the plot is denser close to $y = 0.5$ than at $y=0.8$ or $y=0.2$. }
\end{figure}

\newpage
\begin{figure}[!ht]
\centering
\includegraphics[width=12cm, height=7.5cm]{./Plots/1_hist_uniformnes.pdf}
\caption{The uniforms of the random number generator for 1 million random values. The values are binned in 20 bins. A good random number generator should fluctuate around 50000 $\pm 2\sqrt{50000} = 50000 \pm 447 $ counts per bin (2 sigma). The maximum and minimum amount of counts corresponds to  50444 and 49642 counts. These value's just lay withing the 2 sigma uncertainty.  The uniformness of the random number generator therefore appears to be quite acceptable.  }
\end{figure}
\end{quote}


%\textbf{Code - helper } 
%\begin{quote}
%The code for the Poisson distribution and the factorial function.  
%\lstinputlisting[firstline=2,lastline=46]{./code/mathlib/utils.py}
%\end{quote}


%\textbf{Output}
%\begin{quote}
%The output produced by \textsf{/code/assigment1\_ a.py} 
%\lstinputlisting{./output/assigment1_a_out.txt}
%\end{quote}
















  







\end{document}
