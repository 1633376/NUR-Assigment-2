\documentclass[a4paper,10pt]{article}
\usepackage[utf8]{inputenc}
\input{./Tex/included_packages}

\renewenvironment{abstract}
 { \vspace*{0.3cm} \textbf{\abstractname} \vspace{0.1cm} \\ \ignorespaces}
 {\par\medskip \vspace{0.1cm}}

\setlength{\parindent}{0em}

\setlength{\textheight}{25.7cm}
\setlength{\textwidth}{18cm}
\setlength{\unitlength}{1mm}
\setlength{\topskip}{2truecm}

\topmargin 260mm \advance \topmargin -\textheight
\divide \topmargin by 2 \advance \topmargin -1in
\headheight 0pt \headsep 0pt \leftmargin 210mm \advance
\leftmargin -\textwidth
\divide \leftmargin by 2 \advance \leftmargin -1in
\oddsidemargin \leftmargin \evensidemargin \leftmargin
\parindent=0pt
\frenchspacing


%opening
\title{\textbf{Numerical Recipes for Astrophysics \\ Solutions hand-in assignment-2}}
\author{Luther Algra - s1633376}

\begin{document}

\maketitle
% two segments of code. The first segment of code contains the full code of the program that executes the sub-questions. The second segment contains the shared modules (if any) used by the sub-question.
\hrule
\begin{abstract}
The current document contains the solutions for the second hand-in assignment of Numerical Recipes. The main questions, 1, 2, 3 ..., 7, are in this document all given their own section. Each section contains a subsection for its related sub-questions (1.a, 1.b, 1.c, ..., 1.e) and ends with a final subsection that contains two segments of code. The first segment contains the code for the full main question. The second segment contains the code of shared modules used by the sub-questions. A sub-question itself always starts with a short summary of the question that needs to be answered. The summary is followed by an explanation of how the problem is solved and the code that provides the solution. The output of the code is always presented after the code  and is there discussed if necessary.


\end{abstract}
\hrule
\vspace{0.5cm}


\section*{\textbf{1 - Normally distributed pseudo-random numbers} \hrule} 



\subsection*{\textbf{Question 1.a}}
\begin{quote}

\textbf{Problem}
\begin{quote}Write a random number generator that returns a random floating-point number between 0 and 1. At minimum, use some combination of an MWC and a 64-bit XOR-shift. Plot a sequential of random numbers against each other in a scatter plot ($x_{i+1}$ vs $x_{i}$) for the first 1000 numbers generated. Also plot the value of the random numbers for the first 1000 numbers vs the index of the random number, this mean the x-axis has a value from 0 through 999 and the y-axis 0 through 1). Finally, have your code generate 1,000,000 random numbers and plot the result of binning these in 20 bins 0.05 wide. 
\end{quote}

\textbf{Solution} 


\begin{quote}
The state of the random number generator is updated by first performing a 64-bit XOR-shift on the current state. Next, a modified version of the 64-bit XOR-shift output is given to the MWC algorithm. The modified XOR-shifts output given to the MWC algorithm is the output of the 64 XOR-shift with the last 32 bits put to zero. This is done by performing the 'AND' operation with the maximum value of an unsigned int 32. This modification was performed as the  MWC algorithm expects as input a 64-bit unsigned integer with a value between  $0 < x < 2^{32}$.  The output of the MWC is finally XORd with the unmodified output of the 64-bit XOR-shift. The result is set as new state.
\\
The first 32 bits of the new state are used to provide a random value, as the output of the MWC algorithm only contains 32 significant bits. This random value is obtained by performing the 'AND' operation between the seed and the maximum value of an unsigned int 32. The resulting value is then divided by the maximum value of an uint32 to obtain a value between 0 and 1.
\\
The code for the random number generator can be found at the end of this section, as it is treated as a shared module. The code for generating the plots and the created plots can be found below. The code does not only print the random seed, but also prints the maximum and minimum number of counts for the binned 1,000,000 values. These values are referred to in the description of the plot that displayed the uniformness.
\end{quote}
\newpage

\textbf{Code - Plots}


% consists of the code that initializes the random number generator and calls the function.

\begin{quote}
The code for generating the plots. The used imports and the initialization of the random number generator are not explicit shown in this piece of code, but can be found on page .. where the full code is shown that contains all sub-questions together. The code for the random number generator can be found on page,... as it is treated as a shared module.

\lstinputlisting[firstline=25,lastline=59]{./Code/assigment_1.py}
\end{quote}
\end{quote}

\textbf{Code - Output text } 
\begin{quote}
The text output produced by the code:
\lstinputlisting[firstline=0,lastline=1]{./Output/assigment1_out.txt}
\end{quote}
\newpage

\textbf{Code - Output plots}
\begin{quote}

\begin{figure}[!ht]
\centering
\includegraphics[width=12cm, height=7.5cm]{./Plots/1_plot_against.pdf}
\caption{A plot of random number $x_{i+1}$ against random number  $x_{i}$ for the first 1000 random uniforms produced by the random number generator. A good random number generator should produce a homogeneous plot without many (large) empty spots. In the above plot large empty spots appear to be absent, which suggests that the random number generator passes this test. }
\end{figure}

\begin{figure}[!hb]
\centering
\includegraphics[width=12cm, height=8.0cm]{./Plots/1_plot_index.pdf}
\caption{The first 1000 random uniform numbers produced by the random number generator (RNG) against their index. A good random number generator should not have large wide gaps  (e.g when moving from index 400 to 450 it should not only produce values larger than 0.8, which would leave a wide gap). In the plot these gaps appear to be absent. The average value produced by the RNG should furthermore be 0.5. This corresponds to rapidly moving up and down  the line $ y = 0.5$. In the plot this should, result in a 'dense' region (less white) around the line $y = 0.5$. It can indeed be seen that the plot is denser close to $y = 0.5$ than at $y=0.8$ or $y=0.2$. }
\end{figure}

\newpage
\begin{figure}[!ht]
\centering
\includegraphics[width=12cm, height=7.5cm]{./Plots/1_hist_uniformnes.pdf}
\caption{The uniforms of the random number generator for 1 million random values. The values are binned in 20 bins. A good random number generator should fluctuate around 50000 $\pm 2\sqrt{50000} = 50000 \pm 447 $ counts per bin (2 sigma). The maximum and minimum amount of counts corresponds to  50444 and 49642 counts. These value's just lay withing the 2 sigma uncertainty.  The uniformness of the random number generator therefore appears to be quite acceptable.  }
\end{figure}
\end{quote}


%\textbf{Code - helper } 
%\begin{quote}
%The code for the Poisson distribution and the factorial function.  
%\lstinputlisting[firstline=2,lastline=46]{./code/mathlib/utils.py}
%\end{quote}


%\textbf{Output}
%\begin{quote}
%The output produced by \textsf{/code/assigment1\_ a.py} 
%\lstinputlisting{./output/assigment1_a_out.txt}
%\end{quote}














\subsection*{\textbf{Question 1.b}}
\begin{quote}

\textbf{Problem}
\begin{quote}Now use the Box-Muller method to generate 1000 normally-distributed random numbers. To check if they are following the expected Gaussian distribution, make a histogram (scaled appropriate) with the corresponding true probability distribution (normalized to integrate to 1) as line. This plot should contain the interval of -5 $\sigma$ until $5\sigma$ from the theoretical probability distribution. Indicate the theoretical $1\sigma$, $2\sigma$, $3\sigma$ and $4\sigma$ interval with a line. For this plot, use $\mu =3$ and $\sigma = 2.4$ and choose bins that are appropriate.
\end{quote}

\textbf{Solution} 



\begin{quote}
The Box-Muller method allows two i.i.d uniform variables to be transformed to two i.i.d Gaussian distributed variables. It thereby overcomes the problem that there lacks a closed form for the CDF of a Gaussian distribution. The method overcomes the problem by transforming the joined CDF of the two random Gaussian variables to polar coordinates. This transformation makes it possible to find the CDFs for the polar coordinates of the random Gaussian variables. The CDFs can be used to convert the two uniform distributed variables to the polar coordinates of the Gaussian distributed variables. The polar coordinates can then finally be transformed back to Cartesian coordinates to find the transformation between the two uniform random variables and two gaussian random variables.

 %and then by determing the CDFs of the polar coordinates. In polar coordinates it becomes possible to find the transformation between the uniform distributed random variables and the polar coordinates of the  Gaussian random variables. The polar coordinates can then be converted back to Cartesian coordinates to find the transformation between the two uniform random variables to the two gaussian random variables.


Let $X, Y \sim G(\mu, \sigma ^2)$ be two i.i.d Gaussian distributed random variables. Their joined CDF is then given by, 
\begin{equation}
P(X \leq x_1, Y \leq y_1) =  \int_{-\infty}^{x_1} \int_{-\infty}^{y_1} G(x| \mu, \sigma^2) G(y| \mu, \sigma^2) dx dy
\end{equation}

Transforming to polar coordinates by substituting $ (x-\mu) = r \cos(\theta)$ and $ (y-\mu) = r\sin(\theta)$ yields,

\begin{align*}
P(R \leq r_1, \Theta \leq \theta_1) &= \int_0^{r_1} \int_{0}^{\theta_1} G(r\cos(\theta) \sigma + \mu| \mu, \sigma^2) G(r\sin(\theta) \sigma + \mu| \mu, \sigma^2) r dr d\theta \\
&= \frac{1}{2 \pi \sigma^2} \int_0^{r_1} \int_{0}^{\theta_1} re^{ -\frac{1}{2} \left[ \left( \frac{r\cos(\theta)}{\sigma} \right)^2  + \left( \frac{r\sin(\theta)}{\sigma} \right)^2 \right]}  dr d\theta \\
&=  \frac{1}{2 \pi \sigma^2} \int_0^{r_1} \int_{0}^{\theta_1} re^{ -\frac{r^2}{2 \sigma ^2} } dr d\theta
\end{align*}

The CDF's  for the polar coordinates are now given by, % $R$ and $\theta$ are now given by, 

\begin{align}
P(R \leq r_1) &= \frac{1}{\sigma^2} \int_{0}^{r_1}  re^{ -\frac{r^2}{2 \sigma ^2} } dr =  \int_{0}^{r_1} \frac{d}{dr} \left( -e^{ -\frac{r^2}{2 \sigma ^2}}  \right) dr = 1 - e^{- \frac{r_1^2}{2 \sigma^2}} \\
P(\Theta \leq \theta_1) &= \frac{1}{2 \pi }  \left[ -e^{-\frac{r^2}{2 \sigma^2}} \right]^{\infty}_{0} \int_{0}^{\theta_1} d\theta = \frac{\theta_1}{2\pi}
\end{align}

Let $U_1, U_2 \sim U(0,1)$ be two i.i.d uniform variables. From the transformation law of probability we then must have that, %The transformation law of probability then states that,

\begin{align}
P(R \leq r_1) &= P(U_1 \leq u_1) \rightarrow  1 - e^{- \frac{r_1^2}{2\sigma^2}}  = \int_{0}^{u1} du_1 = u_1 \\
P(\Theta \leq \theta) &= P(U_2 \leq u_2) \rightarrow   \frac{\theta_1}{2\pi} = \int_{0}^{u2} du_2 = u_2 
\end{align}

The transformation from the two uniform distributed variables to the polar coordinates of the Gaussian distributed variables then becomes,

\begin{align}
r_1 &=  \sqrt{-2\sigma^2 \ln(1 - u_1)} \\
\theta_1 &= 2 \pi u_2
\end{align}

Finally converting back to carthesian c oordinates then yields the transformation from two i.i.d uniform distributed variables to two i.i.d gaussian distributed variables;
\begin{align*}
x_1 &= r\cos(\theta) + \mu = \sqrt{-2\sigma^2 \ln(1 - u_1)} \cos( 2 \pi u_2 ) + \mu \\
y_1 &= r\sin(\theta) + \mu = \sqrt{-2\sigma^2 \ln(1 - u_1)} \sin( 2 \pi u_2 ) + \mu
\end{align*}


\end{quote}
\end{quote}

%\textbf{Code - output } 
%\begin{quote}
% The code that produces the output.
%\lstinputlisting{./code/assigment1_a.py}
%\end{quote}

%\textbf{Code - helper } 
%\begin{quote}
%The code for the Poisson distribution and the factorial function.  
%\lstinputlisting[firstline=2,lastline=46]{./code/mathlib/utils.py}
%\end{quote}


%\textbf{Output}
%\begin{quote}
%The output produced by \textsf{/code/assigment1\_ a.py} 
%\lstinputlisting{./output/assigment1_a_out.txt}
%\end{quote}
\newpage













\subsection*{\textbf{Question 1.c}}
\begin{quote}

\textbf{Problem}
\begin{quote}
Write a code that can do the KS-test on the your function to determine if it is consistent with a normal distribution. For this, use $\mu = 0$ and $\sigma = 1$. Make a plot of the probability that your Gaussian random number generator is consistent with Gaussian distributed random numbers, start with 10 random numbers and use in your plot a spacing of 0.1 dex until you have calculated it for $10^5$ random numbers on the x-axis. Compare your algorithm with the KS-test function from \texttt{scipy, scipy.stats.kstest} by making an other plot with the result from your KS-test and the KS-test from scipy.
\end{quote}

\textbf{Solution} 
\begin{quote}
The implementation of the KS-test is in general straight forwards. There are however two points of interest that needs to be discussed. The first point is the implementation of the CDF for the KS-statistic and the second point is the implementation of the CDF for the normal distribution.
\\

\textbf{(1) CDF KS-statistic }
\begin{quote}
The p-value produced by the KS-tests requires the evaluation of the CDF for the KS-test statistic,
\begin{equation}
P_{KS}(z) = \frac{2\sqrt{\pi}}{z} \sum_{j=1}^{\infty} \exp\left(- \frac{(2j-1)^2+\pi^2}{8z^2} \right)
\end{equation}

This infinite sum needs to be numerically approximated in order to perform the KS-test. The chosen approximation in the implementation of the KS-test for  the sum is taken from the book \textit{Numerical Recipes - The art of Scientific Computation, 3d edition}, %... who states that the sum can be approximated by, % approximation can according to .. be approximated by, 
\begin{equation}
P_{KS}(z) \approx
\begin{cases}
\frac{\sqrt{2 \pi}}{z} \left[ \left( e^{-\pi^2/(8z^2)} \right)+^{9} + \left( e^{-\pi^2/(8z^2)} \right) \left( e^{-\pi^2/(8z^2)} \right)^{25} \right] &\quad \text{for $ z < 1.18$}\\
1 -2 \left[ \left(e^{-2z^2}\right) - \left(e^{-2z^2}\right)^4 + \left(e^{-2z^2}\right)^9 \right]  &\quad \text{for $ z >= 1.18$}
\end{cases}
\end{equation}
\end{quote}

\textbf{(2) CDF normal distribution}
\begin{quote}
The CDF of the normal distribution is needed in order to perform the KS-test under the null hypothesis that the data follows a normal distribution. The CDF of the normal distribution can in general be written as,
\begin{equation}
\Phi\left( \frac{x- \mu}{\sigma}\right) = \frac{1}{2} \left[ 1 + \text{erf} \left( \frac{x - \mu}{\sigma\sqrt{2}} \right) \right]
\end{equation}

where the erf is given by,
\begin{equation}
\text{erf}(x)  \frac{2}{\sqrt{\pi}} \int_0^{x} e^{-t^2} dt
\end{equation}

The integral of the erf function lacks a closed form and therefore also needs to be numerically approximated. The chosen approximation is taken from \textit{Abramowitz and Stegun},

\begin{equation}
\text{erf}(x) \approx 1- (a_1t+a_2t^2 + ... + a_5t^5)e^{-x^2} \text{x} \quad t = \frac{1}{1+px}
\end{equation}

where $p  =0.3275911$, $a_1 =  0.254829592$, $a_2 = -0.284496736$, $a_3 = 1.421413741$, $a_4 =  -1.453152027$, $a_5 = 1.061405429$.
\end{quote}

The KS-test and the CDF are implemented with these approximations. The code for the KS-test and the CDF  is located in the file \texttt{./Code/mathlib/statistics.py} at page \pageref{CODE:Statistics}, as this file is threaded as a shared module. The KS-test does require an sorting algorithm, this algorithm is implemented in the file \texttt{./Code/matlib/sorting} and can be found on page \pageref{CODE:sorting}. 
 The code for the generation of the plots and plots are displayed below.  
\end{quote}

\newpage
\textbf{Code - Plots}

\begin{quote}
The code for generating the two plots. The imports for this file are not explicit shown, but can be found on page \pageref{CODE:MAIN1}.
\lstinputlisting[firstline=125,lastline=180]{./Code/assigment_1.py}
\end{quote}
\newpage

\textbf{Code - Output plot(s)}
\begin{quote}
\begin{figure}[!ht]
\centering
\includegraphics[width=12cm, height=7.5cm]{./Plots/1_plot_ks_test_self.pdf}
\caption{The P-value produced by the KS-test against the number of samples on which the KS-test is performed for the self written RNG. The red line indicates the line of $ p = 0.05$. A point \textbf{below} the line would suggests that there is enough statistical evidence to reject the (null) hypothesis that the data is normal distributed. The plot shows that the RNG always passes KS-test up to atleast $10^5$ samples. The p-value  does however appear to drop for a large number of samples and might even drop further when more samples are used. The drop suggests again that the RNG is likely not perfect.}
\end{figure}

\begin{figure}[!hb]
\centering
\includegraphics[width=12cm, height=7.5cm]{./Plots/1_plot_ks_test_self_scipy.pdf}
\caption{The P-value produced by the KS-test against the number of samples on which the KS-test is performed for the self written RNG. The red line indicates the line of $ p = 0.05$. The orange line is the self written implementation of the KS-test and the blue line is the scipy version.  A point \textbf{below} the red line would suggests that there is enough statistical evidence to reject the (null) hypothesis that the data is normal distributed. The self written KS-test is  close to the scipy version, but shows (small) deviations at small sample sizes (for example at $N_{samples} = 10$ or $N_{samples} = 200$).  The self written implementation always has the same shape as the scipy version, even at the deviations. The exact cause for the deviations are unknown, but are likely the result of an approximation that scipy makes that the self written implementation doesn't make. (This is not confirmed by looking at the scipy code.) }
\end{figure}
\end{quote}

\end{quote}



%\textbf{Code - output } 
%\begin{quote}
% The code that produces the output.
%\lstinputlisting{./code/assigment1_a.py}
%\end{quote}

%\textbf{Code - helper } 
%\begin{quote}
%The code for the Poisson distribution and the factorial function.  
%\lstinputlisting[firstline=2,lastline=46]{./code/mathlib/utils.py}
%\end{quote}


%\textbf{Output}
%\begin{quote}
%The output produced by \textsf{/code/assigment1\_ a.py} 
%\lstinputlisting{./output/assigment1_a_out.txt}
%\end{quote}













\subsection*{\textbf{Question 1.c}}
\begin{quote}

\textbf{Problem}
\begin{quote}
Write a code that does the Kuiper’s test on your random numbers (see tutorial
8) and make the same plot as for the KS-test.
\end{quote}

\textbf{Solution} 
\begin{quote}
The implementation of the Kuiper test does simmilar to the KS-test require a numerical approximation of the kuiper statistics,

%KS-test is in general straight forwards. There are however two points of interest that needs to be discussed. The first point is the implementation of the CDF for the KS-statistic and the second point is the implementation of the CDF for the normal distribution.
\\

\textbf{(1) CDF KS-statistic }
\begin{quote}
The p-value produced by the KS-tests requires the evaluation of the CDF for the KS-test statistic,
\begin{equation}
P_{KS}(z) = \frac{2\sqrt{\pi}}{z} \sum_{j=1}^{\infty} \exp\left(- \frac{(2j-1)^2+\pi^2}{8z^2} \right)
\end{equation}

This infinite sum needs to be numerically approximated in order to perform the KS-test. The chosen approximation in the implementation of the KS-test for  the sum is taken from ... who states that the sum can be approximated by, % approximation can according to .. be approximated by, 
\begin{equation}
P_{KS}(z) \approx
\begin{cases}
\frac{\sqrt{2 \pi}}{z} \left[ \left( e^{-\pi^2/(8z^2)} \right)+^{9} + \left( e^{-\pi^2/(8z^2)} \right) \left( e^{-\pi^2/(8z^2)} \right)^{25} \right] &\quad \text{for $ z < 1.18$}\\
1 -2 \left[ \left(e^{-2z^2}\right) - \left(e^{-2z^2}\right)^4 + \left(e^{-2z^2}\right)^9 \right]  &\quad \text{for $ z >= 1.18$}
\end{cases}
\end{equation}
\end{quote}

\textbf{(2) CDF normal distribution}
\begin{quote}
The CDF of the normal distribution is needed in order to perform the KS-test under the null hypothesis that the data follows a normal distribution. The CDF of the normal distribution can in general be written as,
\begin{equation}
\Phi\left( \frac{x- \mu}{\sigma}\right) = \frac{1}{2} \left[ 1 + \text{erf} \left( \frac{x - \mu}{\sigma\sqrt{2}} \right) \right]
\end{equation}

where the erf is given by,
\begin{equation}
\text{erf}(x)  \frac{2}{\sqrt{\pi}} \int_0^{x} e^{-t^2} dt
\end{equation}

The integral of the erf function lacks a closed form and therefore also needs to be numerically approximated. The chosen approximation is taken from ..., who states that the function can be 
approximated by,

\begin{equation}
\text{erf}(x) \approx 1- (a_1t+a_2t^2 + ... + a_5t^5)e^{-x^2} \text{x} \quad t = \frac{1}{1+px}
\end{equation}

where $p  =0.3275911$, $a_1 =  0.254829592$, $a_2 = -0.284496736$, $a_3 = 1.421413741$, $a_4 =  -1.453152027$, $a_5 = 1.061405429$.
\end{quote}

The implementation of the KS-tests and using it to test the null hypothesis that the samples follow a normal distribution is with the above approximations implemented. The code for the KS-test and the approximations for the CDF's is located in the file .... at page, as this file is threaded as module. The code for the creation of the plot is given below.  
\end{quote}

\newpage
\textbf{Code - Plots}

\begin{quote}
The code for generating the two plots.
\lstinputlisting[firstline=117,lastline=169]{./Code/assigment1.py}
\end{quote}
\newpage

\textbf{Code - Output plot(s)}
\begin{quote}
\begin{figure}[!ht]
\centering
\includegraphics[width=12cm, height=7.5cm]{./Plots/1_plot_ks_test_self.pdf}
\caption{TODO}
\end{figure}

\begin{figure}[!hb]
\centering
\includegraphics[width=12cm, height=7.5cm]{./Plots/1_plot_ks_test_self_scipy.pdf}
\caption{TODO}
\end{figure}

\end{quote}



\end{quote}

%\textbf{Code - output } 
%\begin{quote}
% The code that produces the output.
%\lstinputlisting{./code/assigment1_a.py}
%\end{quote}

%\textbf{Code - helper } 
%\begin{quote}
%The code for the Poisson distribution and the factorial function.  
%\lstinputlisting[firstline=2,lastline=46]{./code/mathlib/utils.py}
%\end{quote}


%\textbf{Output}
%\begin{quote}
%The output produced by \textsf{/code/assigment1\_ a.py} 
%\lstinputlisting{./output/assigment1_a_out.txt}
%\end{quote}
\newpage

















  







\end{document}
